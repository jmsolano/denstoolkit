\documentclass[11pt]{amsart}
\usepackage{geometry}                % See geometry.pdf to learn the layout options. There are lots.
\geometry{letterpaper}                   % ... or a4paper or a5paper or ... 
%\geometry{landscape}                % Activate for for rotated page geometry
%\usepackage[parfill]{parskip}    % Activate to begin paragraphs with an empty line rather than an indent
\usepackage{graphicx}
\usepackage{amssymb}
\usepackage{epstopdf}
\DeclareGraphicsRule{.tif}{png}{.png}{`convert #1 `dirname #1`/`basename #1 .tif`.png}
\usepackage{color}


\newcommand{\vecx}{\boldsymbol{x}}
\newcommand{\tnabla}{\tilde{\nabla}}
\newcommand{\ds}{\partial_{s}}
\newcommand{\us}{\hat{s}}
\newcommand{\dt}{\partial_{t}}
\newcommand{\ut}{\hat{t}}
\newcommand{\dA}{\dot{A}}
\newcommand{\dB}{\dot{B}}


\title{Projection of Density Matrix of Order 1 onto a unit square}
\author{J. M. Solano-Altamirano}
\date{\today}                                           % Activate to display a given date or no date

\begin{document}
\begin{abstract}
We present formulae for projecting the Density Matrix of Order 1 onto the unit square. We also
develop expressions for first and second derivatives within the unit square.
\end{abstract}
\maketitle


%++++++++++++++++++++++++++++++++++++++++++++++++++++++++++++++++
\section{Straight line parametrisation}
%++++++++++++++++++++++++++++++++++++++++++++++++++++++++++++++++
%----------------------------------------------------------------
\subsection{Projection}
%----------------------------------------------------------------

In order to project the Density Matrix of Order 1 (hereafter DM1) onto a unit square with
coordinates $(s,t)$, \textit{i.e.}
%
\begin{equation}
   \Gamma(s,t)=\Gamma\left(\vecx(s),\vecx'(t)\right)
\end{equation}
%
we define the following parametrisation
\begin{subequations}\label{eq:parametstraight}
\begin{eqnarray}
\vecx_{s} & \equiv & \vecx_1+(\vecx_2-\vecx_1)s,\\
\vecx'_{t} & \equiv & \vecx'_1+(\vecx'_2-\vecx'_1)t.
\end{eqnarray}
\end{subequations}

Here $s\in[0,1]$, and $t\in[0,1]$; $\vecx_1$, $\vecx_2$, $\vecx'_1$, and $\vecx'_2$ are
four reference points within the real space. Usually these points are selected in such a manner
that they correspond to the coordinates of two nuclei, or two Attractor Critical Points.

Parametrisation (\ref{eq:parametstraight}) only works for a straight line, however, we will begin
with this in order to better understand how to obtain derivatives of the projected DM1.


%----------------------------------------------------------------
\subsection{Derivatives within the unit square}
%----------------------------------------------------------------

Let us define a gradient upon the unit square as
\begin{equation}
   \tnabla\Gamma=\ds\Gamma\us+\dt\Gamma\ut.
\end{equation}

The first derivatives w.r.t. $s$ and $t$ are obtained through the chain rule
(Einstein summation convention is applied):
%
\begin{subequations}
\begin{eqnarray}
   \partial_s\Gamma
   &=&\frac{\partial x^i}{\partial s}\frac{\partial\Gamma}{\partial x^i}
   =(x^i_2-x^i_1)\partial_i\Gamma(x^i,x'^j)
   =c_{\dA\dB}(x^i_2-x^i_1)\phi_{\dB}(\vecx')\partial_i\phi_{\dA}(\vecx)\\
   \partial_t\Gamma
   &=&\frac{\partial x'^j}{\partial t}\frac{\partial\Gamma}{\partial x'^j}
   =(x'^j_2-x'^j_1)\partial_j\Gamma(x^i,x'^j)
   =c_{\dA\dB}(x'^j_2-x'^j_1)\phi_{\dA}(\vecx)\partial'_j\phi_{\dB}(\vecx').
\end{eqnarray}
\end{subequations}
%

Similarly, second derivatives can be obtained, which yields:
%
\begin{subequations}
\begin{eqnarray}
   \partial_s^2\Gamma &=& (x^j_2-x^j_1)(x^i_2-x^i_1)\partial_i\partial_j\Gamma,\\
   \partial_t^2\Gamma &=& (x'^j_2-x'^j_1)(x'^i_2-x'^i_1)\partial_i'\partial_j'\Gamma,\\
   \partial_s\partial_t\Gamma &=& (x'^j_2-x'^j_1)(x^i_2-x^i_1)\partial_i\partial_j'\Gamma,\\
   \partial_t\partial_s\Gamma &=& (x^j_2-x^j_1)(x'^i_2-x'^i_1)\partial_i'\partial_j\Gamma.
\end{eqnarray}
\end{subequations}
%
Note that
%
\begin{equation}
   \partial_t\partial_s\Gamma=\partial_s\partial_t\Gamma
\end{equation}
%
And in terms of primitives, second derivatives are
%
\begin{subequations}
\begin{eqnarray}
   \partial_i\partial_j\Gamma&=&c_{\dA\dB}\phi_{\dB}(\vecx')
      \partial_i\partial_j\phi_{\dA}(\vecx)\\
   \partial_i'\partial_j'\Gamma&=&c_{\dA\dB}\phi_{\dA}(\vecx)
      \partial_i'\partial_j'\phi_{\dB}(\vecx'),\\
      \partial_i\partial_j'\Gamma&=&c_{\dA\dB}\partial_i\phi_{\dA}(\vecx)
      \partial_j'\phi_{\dB}(\vecx')
\end{eqnarray}
\end{subequations}
%

%++++++++++++++++++++++++++++++++++++++++++++++++++++++++++++++++
\section{Generic curve parametrisation}
%++++++++++++++++++++++++++++++++++++++++++++++++++++++++++++++++

%----------------------------------------------------------------
\subsection{Projection}
%----------------------------------------------------------------

When we want to parametrise a generic curve, $c$, we also can map the 
curve in three-dimensional space to a square, and even to a unit square. The projection becomes
slightly more complicated, unless we know an analytic expression for
obtaining the curve $c$ as a function of a parameter $s$, but we
still have a relatively easy procedure.

For definiteness and simplicity, here we parametrise a curve in three-dimensional space
comprised by a finite set of points (see Fig. \ref{fig:arbcurveparam}), and we will map
the curve $c$ onto a line, $s$ of length $|c|$, \textit{i.e.}, the line $s$ has the same
length as the curve in three dimensional space.

\begin{figure}\label{fig:arbcurveparam}
\centering
\includegraphics[width=0.75\textwidth]{plots/CurveParam01.pdf}
\caption{Parametrisation of an arbitrary curve (compised by a finite
set of points).}
\end{figure}

At each of the points $c_i$ we know the corresponding three-dimensional coordinates,
therefore a function, $f$, takes the following values
\begin{equation}
   f(\vecx_i)=f(c_i)=f(s_i).
\end{equation}

First derivatives may be evaluated as follows:
%
\begin{equation}
\left.\frac{\partial f}{\partial c}\right|_i=
\left(\left.\frac{\partial x^j}{\partial c}\right|_i\right)
\frac{\partial f}{\partial x^j}
\end{equation}
%

For determining the first derivatives at each of the known points, we can use a finite-difference
approach, hence (we assume we know $N$ points of the curve $c$)
%
\begin{equation}
\left.\frac{\partial x^j}{\partial c}\right|_i=\left\{
\begin{array}{ll}
   \frac{\left(x^j_{i+1}-x^j_i\right)}{|\vecx_{i+1}-\vecx_i|} & i = 0,\dots,N-1 \\ \\
   -\frac{\left(x^j_{N}-x^j_(N-1)\right)}{|\vecx_{N}-\vecx_{N-1}|} & i = N
\end{array}
\right.
\end{equation}
%

In the first implementation of this algorithm in DensToolKit, we will use the difference between
the length of the bond path, $b$  and the line that joins the two ACPs, $ACP_1$ and $ACP_2$, associated to the corresponding BCP as a validation parameter, and we will accept as valid the above approximation whenever we have
$b<1.25|\vecx_{ACP_1}-\vecx_{ACP_2}|$.

%----------------------------------------------------------------
\subsubsection{Numerical second derivatives, variable steps}
%----------------------------------------------------------------

Second derivatives may be numerically obtained by using the definition of the
first derivative, \textit{i.e.}
%
\begin{eqnarray}%\label{eq:}
  f''(x) & = &\frac{f'(x+h)-f'(x)}{h}\nonumber\\
  & = &\frac{1}{h}\left\{ \frac{f(x+h+g)-f(x+h)}{g}-\frac{f(x+h)-f(x)}{h} \right\}\nonumber\\
  & = & \frac{hf(x+h+g)-(h+g)f(x+h)+gf(x)}{gh^2}.
\end{eqnarray}
%


\end{document}  
