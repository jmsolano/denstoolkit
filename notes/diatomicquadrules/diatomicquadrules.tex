\documentclass[11pt]{amsart}
\usepackage{geometry}                % See geometry.pdf to learn the layout options. There are lots.
\geometry{letterpaper}                   % ... or a4paper or a5paper or ... 
%\geometry{landscape}                % Activate for for rotated page geometry
%\usepackage[parfill]{parskip}    % Activate to begin paragraphs with an empty line rather than an indent
\usepackage{amsmath}
\usepackage{graphicx}
\usepackage{amssymb}
\usepackage{epstopdf}
\DeclareGraphicsRule{.tif}{png}{.png}{`convert #1 `dirname #1`/`basename #1 .tif`.png}
\usepackage{color}


\newcommand{\vecx}{\boldsymbol{x}}
\newcommand{\tnabla}{\tilde{\nabla}}
\newcommand{\ds}{\partial_{s}}
\newcommand{\us}{\hat{s}}
\newcommand{\dt}{\partial_{t}}
\newcommand{\ut}{\hat{t}}
\newcommand{\dA}{\dot{A}}
\newcommand{\dB}{\dot{B}}


\title{Diatomic cubature rules}
\author{J. M. Solano-Altamirano}
\date{\today}                                           % Activate to display a given date or no date

\begin{document}
\begin{abstract}
We describe how to construct cubature rules for diatomic molecules. In the current version,
atoms are required to be located along the $z$-axis for the rules to work properly.
\end{abstract}
\maketitle


%++++++++++++++++++++++++++++++++++++++++++++++++++++++++++++++++
\section{Core cubature}
%++++++++++++++++++++++++++++++++++++++++++++++++++++++++++++++++
%----------------------------------------------------------------
\subsection{Upper domain}
%----------------------------------------------------------------

If the integration domain is an off-centred sphere of radius $a$, whose centre is displaced along
the global $z$-axis, denoted here as $\Sigma_0$, then
the integral of a azimuthally symmetric function can be reduced to twice the integral
over the two dimensional domain $\Gamma_0$, defined as (see Fig.~\ref{fig:offcentsphdom}
for its geometry):
\begin{equation}
	\Gamma_0\equiv\left\{
		(\rho,\phi) |a\leq\rho\leq0,0\leq\phi\leq\pi
	\right\}.
\end{equation}
%
\textit{i.e.,}
\begin{eqnarray}
	\int_{\Sigma_0}f(r,\theta,\varphi)dV=4\pi\int_0^{\pi}\sin\phi d\phi\int_0^a%
	\rho^2f(\rho,\phi)d\rho\equiv4\pi I.
\end{eqnarray}

\begin{figure}[htb!]
   \centering
   \includegraphics[width=0.4\textwidth]{images/numIntegzoffcentcav  }
   \caption{Scheme of a positively $z$-translated spherical domain, relative to the origin.
   The domain is rotated and translated in such a manner that the spherical domain is
   aligned along the $z$-axis. The projection of the system onto the global 
   $x$-$z$ plane. $a$ is the radius of the spherical domain and $d_u$ is the distance
   between the upper domain centre and the origin of coordinates. The variables
   $(r,\theta)$ and $(\rho,\phi)$ are the polar and radial and polar angles
   relative to the local (primed) and global (non-primed) coordinate systems, respectively.}
   \label{fig:offcentsphdom}
\end{figure}

The numerical scheme (cubature) can be build applying the following change of variables:
\begin{equation}
	p=\frac{2\rho}{a}-1
\end{equation}
and
\begin{equation}
	q=\cos\phi.
\end{equation}
After applying the changes of variables we get:
\begin{equation}
	I=\int_{-1}^1dq\int_{-1}^1\frac{a^3}{8}(p+1)^2f(p,q).
\end{equation}

The cubature rules can be constructed using standard Gauss-Legendre quadrature rules for
$p$ and $q$, \textit{i.e.}, the cubature rules for $I$ are:
%
\begin{equation}
	I=\sum_{i,j}W_iw_jf(p_i,q_j),
\end{equation}
%
where
%
\begin{equation}
	W_i\equiv\left(w_i\frac{a^3}{8}(p_i+1)^2\right),
\end{equation}
%
and $w_i$ ($p_i$) and $w_j$ ($q_j$) are the standard Gauss-Legendre weights (abscissas) related to
$p$ and $q$, respectively.

Finally, the global spherical coordinates are given by:
%
\begin{subequations}
\begin{eqnarray}
	r^2_{ij} & = & \frac{a^2}{4}(p_i+1)^2+a(p_i+1)q_jd_u+d_u^2,\\
	\varphi_{ij} & = & 0,\\
	\theta_{ij} & = & \arcsin\left(\frac{a(p_i+1)}{2r_{ij}}\sqrt{1-q_j^2}\right),
\end{eqnarray}
\end{subequations}
%
and the global Cartesian coordinates are:
%
\begin{subequations}
\begin{eqnarray}
	x_{ij} & = & \frac{a}{2}(p_i+1)\sqrt{1-q_j^2},\\
	y_{ij} & = & 0,\\
	z_{ij} & = & \frac{a}{2}(p_i+1)q_j+d_u.
\end{eqnarray}
%
\end{subequations}

%----------------------------------------------------------------
\subsection{Lower domain}
%----------------------------------------------------------------

The lower cubature rules are obtained by setting $d_u\rightarrow -d_l$,
$a\rightarrow b$, and $z\rightarrow-z$. Here $b$ is the radius of the lower
spherical region of integration, and the centre of it is located at $z=-d_l$.
The weights are the same as for the upper domain, the global spherical coordinates
are:
%
\begin{subequations}
\begin{eqnarray}
	r^2_{ij} & = & \frac{b^2}{4}(p_i+1)^2-b(p_i+1)q_jd_l+d_l^2,\\
	\varphi_{ij} & = & 0,\\
	\theta_{ij} & = & \arcsin\left(\frac{b(p_i+1)}{2r_{ij}}\sqrt{1-q_j^2}\right),
\end{eqnarray}
\end{subequations}
%
and the global Cartesian coordinates are:
%
\begin{subequations}
\begin{eqnarray}
	x_{ij} & = & \frac{b}{2}(p_i+1)\sqrt{1-q_j^2},\\
	y_{ij} & = & 0,\\
	z_{ij} & = & -\frac{b}{2}(p_i+1)q_j-d_l.
\end{eqnarray}
%
\end{subequations}


\end{document}  
