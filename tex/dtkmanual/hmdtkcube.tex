Usage:

	dtkcube wf?name [option [value(s)]] ... [option [value(s)]]

Where wf?name is the input wfx(wfn) name, and options can be:
Where wf?name is the input wfx(wfn) name, and options can be:

  -l        	Writes cpu time, input/output information etc. on a log file
  -n  dim   	Sets the number of points per direction for the cube
            	  to be dim x dim x dim.
  -N nx ny nz	Sets the individual points per direction for the cube
            	  to be nx x ny x nz.
  -o outname	Sets the output file name.
  -s        	Uses a smart cuboid for the grid. The number of points for the
            	  largest direction will be 80.
  -S ln     	Uses a smart cuboid for the grid. ln is the number of points
            	  the largest axis will have. The remaining axes will have
            	  a number of points proportional to its length.
  -c a1 a2  	Uses a cube centred at the midpoint of the atoms a1 and a2.
  -p prop	Chooses the property to be computed. prop is a character,
         	  which can be (d is the default value): 
         		d (Density)
         		g (Magnitude of the Gradient of the Density)
         		l (Laplacian of density)
         		K (Kinetic Energy Density K)
         		G (Kinetic Energy Density G)
         		e (Ellipticity)
         		E (Electron Localization Function -ELF-)
         		L (Localized Orbital Locator -LOL-)
         		M (Magnitude of the gradient of LOL)
         		P (Magnitude of Localized Electrons Detector -LED-)
         		r (Region of Slow electrons -RoSE-)
         		s (Reduced Density Gradient -s-)
         		S (Shannon Entropy Density)
         		V (Molecular Electrostatic Potential)
         		u (Scalar Custom Field)
         		v (Virial Potential Energy Density)
         		Z (Non Covalent Interactions(NCI) -- s)
         		z (Non Covalent Interactions(NCI) -- Rho)
  -P     	Generates a VMD script to render the field. Notice: this requires VMD and the internal
         	  Tachyon render (usually it is included automatically in VMD) to be installed
         	  in your system. The VMD script should run with 'vmd -e filename.vmd'
  -q     	Makes the vmd script to close VMD automatically, after rendering the field.
  -z     	Compresses the cube file using gzip (which must be installed
         	   in your system).
  -V        	Displays the version of this program.
  -h		Display the help menu.

  --configure-nci rMin rMax sMax 	Set the parameters rhoMin, rhoMax,
             		  and redGradMax to be rMin, rMax, and sMax, respectively.
             		  This option only affects NCI cubes (see properties z and
             		  Z in "-p" option. Default values: rhoMin=0.0005,
             		  rhoMax= 0.065, and redGradMax= 2
  --help    		Same as -h
  --version 		Same as -V
