

Usage:

	dtkdemat1 wf?name [option [value(s)]] ... [option [value(s)]]

Where wf?name is the input wfx(wfn) name, and options can be:
Where wf?name is the input wfx(wfn) name, and options can be:

  -a a1 a2  	Define the atoms  (a1,a2) used to define bond path/line.
            	  If this option is not activated, the program will 
            	  set a1=1, a2=2.
            	  Note: if the *.wfn (*.wfx) file has only one atom
            	  the program will exit and no output will be generated.
  -L        	Calculate MD1 upon the straight line that joins the atoms
            	  instead of upon the bond path.
  -n  dim   	Set the number of points for the tsv file per direction.
            	  Note: for the bond path you may want to look for a good 
            	  combination of n and the number "step" given in option -s,
            	  since the number of points in the bond path will be mainly 
            	  governed by step.
  -p prop   	Choose the property to be computed. prop is a character,
            	  which can be (D is the default value):
            	      D (Density Matrix of order 1)
            	      G (Gradient of the density Matrix of order 1)
  -o outname	Set the output file name.
            	  (If not given the program will create one out of
            	  the input name; if given, the tsv, gnp and pdf files will
            	  use this name as well --but different extension--).
  -s step   	Set the stepsize for the bond path to be 'step'.
            	  Default value: 0.03
  -T        	Perform the Topological analysis (find critical points).

  -P        	Create a plot using gnuplot.
  -c        	Show contour lines in the plot.
  -C s i e  	Set the contour values (incremental style).
            	  s, i, and e are real numbers. s is the first
            	  contour value, i is the increment, and 
            	  e is the last contour value.
  -l     	Show labels of atoms (those set in option -a) in the plot.

  -v     	Verbose (display extra information, usually output from third-
         	  party sofware such as gnuplot, etc.)
  -z     	Compress the tsv file using gzip (which must be intalled
         	   in your system).

  -h     	Display the help menu.
  -V     	Displays the version of this program.
  --help    		Same as -h
  --version 		Same as -V

********************************************************************************
  Note that the following programs must be properly installed in your system:
                                    gnuplot
                                    epstool
                                    epstopdf
                                      gzip
********************************************************************************
