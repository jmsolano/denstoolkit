
\newcommand{\wfexfile}{\texttt{ci\-clo\-pro\-pa\-no\_\-pbe\-63\-11.wfx}}
\newcommand{\progusg}[2]{\phantom{asd}\\\texttt{\phantom{MMM}\$#1 ciclopropano\_pbe6311.wfx #2}\\\phantom{asfd}\\}

%**********************************************************************************************
%**********************************************************************************************
\chapter{Programs}
%**********************************************************************************************
%**********************************************************************************************

In this chapter, a description for each program of \DTK{} is provided. The description given here is a bit more detailed than the description offered in each individual help menu of the programs.

Whenever possible, we will provide an example of how to run the program, and after \S\ref{sec:commonopts} it will be assumed that a file called
\wfexfile{} (which should be included along with the source code) exists and is the main input of the program under review.

%----------------------------------------------------------------------------------------------
\section{General behaviour}
%----------------------------------------------------------------------------------------------

%..............................................................................................
\subsection{Input files}
%..............................................................................................

All the programs belonging to \DTK{} read wave function files. These files can be obtained by Computational Quantum Chemistry packages such as Gaussian~\cite{bib:gaussian09}, GAMESS~\cite{bib:gamess}, or NWCHEM~\cite{bib:nwchem}. \DTK{} can read either the old \texttt{wfn} files or the newer \texttt{wfx} files~\cite{bib:webwfxformat}.\footnote{Some incompatibilities may arise when mixing files between Windows and Unix-like operative systems. Please ensure that the files \texttt{wfx} or \texttt{wfn} use the same newline as the operative system where \DTK{} is running. For this, you can use the programs \texttt{dos2unix} and \texttt{unix2dos}.}

The wave function file is \textit{always} the first argument in the command line. The only exception is when requesting the displaying of the help menu (which actually does not evaluate anything). Thus, the usage of every program in \DTK{} obeys the following pattern:\\\phantom{as}\\
\phantom{MMM}\texttt{\$dtk*** inputname.wf? [option [value(s)]] \dots [option [value(s)]]}\\
\phantom{a}\\
Here, the stars represent any of the programs in \DTK, the question mark can be \texttt{n} or \texttt{x}, and the options (with possible values) are specific to each program.

%..............................................................................................
\subsection{Output names}
%..............................................................................................

Most of the programs included in \DTK{} produce some kind of output. The main output is numerical, and it is saved in different files. While the specific output depends on the particular program, there are a few conventions that we adopt. 

\begin{itemize}
   \item \texttt{\textbf{dat files:}} This type contains single or multi-column data and is used whenever a field is evaluated at a line of points (one-dimensional grid). In some output files of this kind, comments are added at the beginning of the file. These comments start with the character \texttt{\#}.
   \item \texttt{\textbf{tsv files:}} This file contains the information for two-dimensional grids. It is the format known as \textit{tab separated values}, and the files also may contain comments starting with the character \texttt{\#}.
   \item \texttt{\textbf{cub files:}} This file contains the information for three-dimensional grids. The format of these files is the standard \texttt{cub} format of Gaussian \cite{bib:gaussian09}.
   \item \texttt{\textbf{gnp files:}} These files are scripts to be parsed to gnuplot. They purpose is to provide basic scripts with non-trivial information about the system that is being plotted. For instance, the positions of labels in the plot, values for the parametrization when lines or planes are projected, colors for the atoms etc.
   \item \texttt{\textbf{pov files:}} These type of files are the input for the pov ray-tracer.
   \item \texttt{\textbf{cpx files:}} This type of file is a native file of \DTK. It contains the information of critical points. These files act both as output and as input. For more details about this format, see \S\ref{sec:cpxfilefmt}.
\end{itemize}

%----------------------------------------------------------------------------------------------
\section{Common options}\label{sec:commonopts}
%----------------------------------------------------------------------------------------------

All programs in \DTK{} share the following options:

\begin{itemize}
   \item \textbf{Help menu:} Usage and some details of the program at hand is displayed by means of\\
   \phantom{MMM}\texttt{\$dtk*** -h}\\or\\
   \phantom{MMM}\texttt{\$dtk*** {-}-help}
   \item \textbf{Version:} The version number can be accessed by\\
   \phantom{MMM}\texttt{\$dtk*** -V}\\or\\
   \phantom{MMM}\texttt{\$dtk*** {-}-version}
   \item\textbf{Output name:} By using this option, the user can give a name other than the automatic name generated by each program. The program at hand will invariably add the appropriate extension to the file-name, and any extension will be consdered part of the name.
   This option is particularly useful when one wishes to give a wfx(wfn) and an out files located at arbitrary directories. For instance, the command\\\phantom{adf}\\
   \phantom{MMM}\texttt{\$dtk*** /arbitrary/path/input.wfx -o /other/path/othername ...}\\\phantom{adf}\\
   will invoke the program \texttt{dtk***} with an input wave function file located at \texttt{/arbit\-rary\-/path} and the output will be saved in the directory \texttt{/other/path} using the name \texttt{othername} with different extensions for the output.
   \item \textbf{Plotting the data:} Most of the \DTK's programs have the capability to create scripts that can be passed to plotting/rendering programs such as gnuplot or povray. While the primary purpose of \DTK{} is to evaluate numerical functions, we are aware that the visualization of the data is also important. With this in mind, and recalling that it is quite common the use of severs dedicated to Molecular Quantum Mechanical calculations, we have found that it is really convenient to be able to produce the visualization of data in a simple manner. If the program has the capability of calling plotting/rendering programs, such a call is activated with\\
   \phantom{MMM}\texttt{\$dtk*** \dots\ -P}\\
   The programs that produce scripts (and internal callings) are: \texttt{dtkline, dtkplane, dtkfindcp, dtkmomd, dtkdemat1, dtkbpdens, dtkqdmol}
   \item \textbf{Compressing data:} Some programs have the option to call \texttt{gzip}. If this option is available, it can be activated with the following command line:\\
     \phantom{MMM}\texttt{\$dtk*** \dots\ -z}\\
     The programs that provide this option are: \texttt{dtkline, dtkplane, dtkcube, dtkmomd}
\end{itemize}

%----------------------------------------------------------------------------------------------
\section{dtkpoint}
%----------------------------------------------------------------------------------------------

This program calculates all the implemented properties that can be calculated at a single point (see \S \ref{sec:availablefields}). Additionally, the program will perform the integration of the electron density over the whole space (which should render the molecule's total number of electrons). The program will take the wave function file and calculate the fields at the point given by the user. The point(s) for the fields can be given in three different forms:
\begin{itemize}
   \item \textbf{Cartesian coordinates}. The fields are calculated at the point $(x,y,z)$. Example:\\
      \progusg{dtkpoint}{-c 0.75e0 -5.0e-01 1.2e0}
   \item \textbf{Set of Cartesian points.} The fields are calculated at a set of points given in an input file. For example, if the file \texttt{coords.dat} contains the coordinates, then the syntax is:\\
      \progusg{dtkpoint}{-i coords.dat}
      
      The file \texttt{coords.dat} contains the follwing data:
      \begin{verbatim}
#Comments accepted by dtkpoint
1.1e0 2.04e-2 1.45e+1
2.3e0 -4.4e-2 3.31543e+00
#Another comment between coordinates
2.3e0 4.4e-2 -3.31543e+00
      \end{verbatim}
      Every line started with the character ``\texttt{\#}'' will be ignored, and comments can be at any line of the input file. This program does not perform any check on the coordinates input file, therefore the user is responsible for providing a file that has $3N$ numbers, where $N$ is the number of non-commented rows in the file. Each row corresponds to a point in the three dimensional space.
      \item \textbf{At a nucleus.} The fields are calculated at the position of a nucleus. Example:\\
         \progusg{dtkpoint}{-a 2}\\
   This will calculate the fields at the position of the 2$^{\textrm{nd}}$ nucleus listed in \wfexfile.
\end{itemize}


\rule{\textwidth}{1pt}
{\center\texttt{dtkpoint} help menu.\\}
\rule{\textwidth}{1pt}
\begin{small}
\begin{verbatim}
Usage:

     dtkpoint wf?name [option [value(s)]] ... [option [value(s)]]

Where wf?name is the input wfx(wfn) name, and options can be:

  -a a         Calculate the properties at the coordinates ofthe a-th atom.
  -c x y z     Calculate the properties at the point (x,y,z).
  -i crdfname  Calculate the properties at all points contained
                 in the file crdfname. The file must be a list of three 
                 space-separated real numbers per row.
                 Please ensure the file has always the format specified above,
                 otherwise unpredictable errors may occur.
  -o outname   Set the output file name.
                 (If not given the program will create one out of
                 the input name.)
  -V           Displays the version of this program.
  -h           Display the help menu.

  --help          Same as -h
  --version       Same as -V
\end{verbatim}
\end{small}
\rule{\textwidth}{1pt}

%----------------------------------------------------------------------------------------------
\section{dtkline}
%----------------------------------------------------------------------------------------------
This program calculates one of the implemented scalar fields (see \S\ref{sec:availablefields}) along a line defined by two of the molecule's atoms. For instance, the ELF along the line that passes through the atoms labeled as 2 and 4 of the cyclopropane molecule (Fig. \ref{fig:cyclopmolline}) is evaluated by means of\\
\progusg{dtkline}{-a 2 4 -p E -P}
Here, the option \texttt{-a} tells \texttt{dtkline} to use the atoms \texttt{2} and \texttt{4} to define the line whereat the property will be evaluated; the option \texttt{-P} tells \texttt{dtkline} to create a pdf image using gnuplot; and with option \texttt{-p E}, one chooses the field ELF (see the help menu of \texttt{dtkline} for a complete list of fields ---below---). The final result, after some editing of the gnuplot script is presented in Fig. \ref{fig:cyclopelfline}.\footnote{Let us recall that \DTK{} facilitates the creation of quality-plots (for instance, by providing the projected coordinates of the atoms C$_2$ and H$_4$ along the line), but it cannot replace the human judgement for creating nice plots.}
\begin{figure}[hb!]
\centering
\subfigure[Cyclopropane molecule.]{\includegraphics[width=0.45\textwidth]{cyclopQDMolLine}\label{fig:cyclopmolline}}\quad%
\subfigure[ELF along the red line of Fig. \ref{fig:cyclopmolline}.]{\includegraphics[width=0.45\textwidth]{cyclopELFLine}\label{fig:cyclopelfline}}
\caption{Evaluating the ELF along the line that joins the atoms C$_2$ and H$_4$ with \texttt{dtkline}.}\label{fig:dtklineuseex}
\end{figure}

Sometimes, it is convenient to increase/decrease the number of points to evaluate within the line. To do so, one can use the option \texttt{-n dim}, where \texttt{dim} is the new number of points. If this option is not especified, \texttt{dtkline} uses 200 points.

\textbf{Special cases:}
\begin{itemize}
   \item \textbf{Wave function with one single atom:} In this case, \texttt{dtkline} will use the single atom present in the wave function file and then define a line that passes through this atom. The option \texttt{-a} is ignored, thus it is not necessary to use it, however if used the syntax must include two numbers. It is recommended, when you are interested in single atom wave functions, better not use the option \texttt{-a}.
   \item \textbf{Wave function with only two atoms:} For diatomic systems, \texttt{dtkline} will use these atoms to define the line, and if option \texttt{-a} is used, it will be ignored. 
\end{itemize}

\rule{\textwidth}{1pt}
{\center\texttt{dtkline} help menu.\\}
\rule{\textwidth}{1pt}
\begin{footnotesize}
\begin{verbatim}
Usage:

	dtkline wf?name [option [value(s)]] ... [option [value(s)]]

Where wf?name is the input wfx(wfn) name, and options can be:

  -a a1 a2     Define the atoms  (a1,a2) used to define the line.
            	  If this option is not activated, the program will 
            	  define the line using the first atom and the vector
            	  (1,1,1).
  -n  dim      Set the number of points for the dat file
  -o outname   Set the output file name.
                 (If not given the program will create one out of
                 the input name; if given, the gnp file and the pdf will
                 use this name as well --but different extension--).
  -p prop      Choose the property to be computed. prop is a character,
                 which can be (d is the default value): 
                   d (Density)
                   g (Magnitude of the Gradient of the Density)
                   l (Laplacian of density)
                   K (Kinetic Energy Density K)
                   G (Kinetic Energy Density G)
                   E (Electron Localization Function -ELF-)
                   L (Localized Orbital Locator -LOL-)
                   M (Magnitude of the Gradient of LOL)
                   S (Shannon Entropy Density)
                   V (Molecular Electrostatic Potential)
  -P        Create a plot using gnuplot.
  -k        Keeps the *.gnp file to be used later by gnuplot.
  -z        Compress the cube file using gzip (which must be intalled
               in your system).
  -V        Displays the version of this program.
  -h        Display the help menu.
  --help       Same as -h
  --version    Same as -V
********************************************************************************
  Note that the following programs must be properly installed in your system:
                             gnuplot,epstopdf,gzip
********************************************************************************
\end{verbatim}
\end{footnotesize}
\rule{\textwidth}{1pt}

%----------------------------------------------------------------------------------------------
\section{dtkplane}
%----------------------------------------------------------------------------------------------

As the name suggests, this program evaluates a field upon a plane. \texttt{dtkplane} defines
a plane based on the choosing of three atoms. Once the atoms are given, the program translates the plane until the centroid of the triangle that the three atoms form coincides with the center of a square. The size of the plane is calculated such that all the atoms in the molecule are contained in the plane, even if not all the atoms are within the plane. Since there is no restriction about the distance between the chosen atoms, it is assumed that the user is interested in studying the field around those atoms, therefore, the produced script only contains one parameter (called ``dimparam'') to zoom in the plotted region.

As an example, the command line\\
\progusg{dtkplane}{-a 1 2 3 -p L -P -k -c -l}
is the base for producing the figure \ref{fig:cycloplolplane}. 
%
\begin{figure}[h!]
\centering
\subfigure[Cyclopropane molecule.]{\includegraphics[width=0.35\textwidth]{cyclopQDMolPlane}\label{fig:cyclopmolplane}}\qquad\qquad%
\subfigure[LOL evaluated at the blue plane of Fig. \ref{fig:cyclopmolplane}.]{\includegraphics[width=0.45\textwidth]{cyclopLOLPlane}\label{fig:cycloplolplane}}
\caption{Evaluating the LOL within the plane that contains the atoms C$_1$, C$_2$, and C$_3$ with \texttt{dtkplane}. The plot is rendered by gnuplot.}\label{fig:dtkplaneuseex}
\end{figure}
%

In the above command line, the option \texttt{-a 1 2 3} is used to provide the atoms that the plain will contain. \texttt{-p L} is the option for choosing the LOL field (see the menu help, below, for the complete list of fields); \texttt{-P} is for rendering a plot by calling gnuplot; \texttt{-k} for keeping the the gnuplot script on disk; \texttt{-c} activates the contours on the plane; and \texttt{-l} activates the labels of the atoms.

\texttt{\textbf{-n}:} For increasing/decreasing the number of points per direction, one can use the option \texttt{-n dim}. This will create a plane with \texttt{dim x dim} points.

\texttt{\textbf{-c:}} This option will print contour lines in the final plot.

\texttt{\textbf{-l:}} To add the labels of the atoms that were used to define the plane.

\texttt{\textbf{-L:}} This option will display the labels of every atom present in the wave function file. The positions are the projected position in the plane.

\texttt{\textbf{-v}:} This option will display information from third party programs such as gnuplot, gzip, etc.

\textbf{Special cases:}
\begin{itemize}
   \item \textbf{Wave function with one single atom:} In this case, \texttt{dtkplane} will use the single atom present in the wave function file and then define a plane that passes through this atom. The option \texttt{-a} is ignored, thus it is not necessary to use it, however if used the syntax must include three numbers. It is recommended, when you are interested in single atom wave functions, better not use the option \texttt{-a}.
   \item \textbf{Wave function with only two atoms:} For diatomic systems, \texttt{dtkplane} will use these atoms to define the plane, and if option \texttt{-a} is used, it will be ignored, however it must contain three numbers.
\end{itemize}

\rule{\textwidth}{1pt}
{\center\texttt{dtkplane} help menu.\\}
\rule{\textwidth}{1pt}
\begin{footnotesize}
\begin{verbatim}
Usage:

      dtkplane wf?name [option [value(s)]] ... [option [value(s)]]

Where wf?name is the input wfx(wfn) name, and options can be:

  -a a1 a2 a3  Define the atoms  (a1,a2,a3) used to define the plane.
                 If this option is not activated, the program will 
                 choose a default plane, but you may not like the view.
                 Note: if the *.wfn (*.wfx) file has only one or two atoms
                 this option must not be used. The program will define
                 a plane which includes that(those) one(two) atom(s).
  -n  dim      Set the number of points for the tsv file per direction
  -o outname   Set the output file name.
                 (If not given the program will create one out of
                 the input name; if given, the tsv, gnp and pdf files will
                 use this name as well --but different extension--).
  -p prop     Choose the property to be computed. prop is a character,
                 which can be (d is the default value): 
                    d (Density)
                    g (Magnitude of the Gradient of the Density)
                    l (Laplacian of density)
                    K (Kinetic Energy Density K)
                    G (Kinetic Energy Density G)
                    E (Electron Localization Function -ELF-)
                    L (Localized Orbital Locator -LOL-)
                    M (Magnitude of the Gradient of LOL)
                    N (Gradient of LOL)
                    S (Shannon Entropy Density)
                    V (Molecular Electrostatic Potential)
  -P          Create a plot using gnuplot.
  -c          Show contour lines in the plot.
  -k          Keeps the *.gnp file to be used later by gnuplot.
  -l          Show labels of atoms (those set in option -a) in the plot.
  -L          Show the labels of ALL of the atoms in the wf? file.
  -v          Verbose (display extra information, usually output from third-
                 party sofware such as gnuplot, etc.)
  -z          Compress the tsv file using gzip (which must be intalled
                 in your system).
  -V          Displays the version of this program.
  -h          Display the help menu.
  --help         Same as -h
  --version      Same as -V
********************************************************************************
  Note that the following programs must be properly installed in your system:
                       gnuplot, epstool, epstopdf, gzip
********************************************************************************
\end{verbatim}
\end{footnotesize}
\rule{\textwidth}{1pt}
%----------------------------------------------------------------------------------------------
\section{dtkcube}
%----------------------------------------------------------------------------------------------

This program evaluates a scalar field within a three dimensional grid. The data is saved in a *.cub file, which is the standard cube file from the Gaussian~\cite{bib:gaussian09} package. This is perhaps the most computationally expensive program in \DTK{}, and also one of the few programs that does not produce any visualization.

The purpose of this program is to create standard output that can be processed by advanced visualization programs such as VMD. For instance, with the command line\\
\progusg{dtkcube}{-p l -S 100}
we produce the cub file that later is used in VMD to create the Figure \ref{fig:dtkcubeuseex}.
%
\begin{figure}[hb!]
\centering
\includegraphics[width=0.45\textwidth]{cyclopLap}\label{fig:cyclopLap}
\caption{Evaluating $\nabla^2\rho$ within a three-dimensional grid (cube). The plot was generated using VMD, and later povray.}\label{fig:dtkcubeuseex}
\end{figure}
%
The option \texttt{-p l} tells \texttt{dtkcube} to evaluate the Laplacian of the electron density (see below for the complete list of available fields), and \texttt{-S 100} activates a \textit{smart} cube. By smart cube, we mean that we try to minimize the number of trivial points, this is, we adjust the cube to be a rectangular cuboid such that it encloses the molecule and a few extra space around the most outer atoms of the molecule. This allows \DTK{} to skip the calculation of spatial points that offer only trivial information (since the wave function and its derivatives decay rapidly as the point moves away from the molecule). For instance, the cuboid generated by the example command generates a grid of \texttt{96 x 100 x 90 = 864,000} points, which is 86.4 \% of the points that would have been calculated if a normal cube had been evaluated.

\texttt{dtkcube} always creates the grid around the geometric center of the molecule. This also reduces the number of trivial points in the final cube/cuboid.

If you want to save the information about wfx input file name, CPU time (time taken to evaluate the whole grid), and some other extra-information, you may want to activate option \texttt{-l}.

%..............................................................................................
\subsection{Managing the dimensions of the grid}
%..............................................................................................

\texttt{dtkcube} offers several options for custom requests:
\begin{itemize}
   \item \textbf{Default cube: } A default centered cube of \texttt{80 x 80 x 80} points is evaluated if no additional options are given.
  	\item\texttt{\textbf{-n dim}}: A centered cube of \texttt{dim x dim x dim} points is evaluated.
	\item\texttt{\textbf{-N nx ny nz}}: A centered cube of \texttt{nx x ny x nz} points is evaluated.
	\item\texttt{\textbf{-s}}: A \textit{smart} cuboid with the largest dimension being \texttt{80} is evaluated. The other dimensions are proportional to the molecule's dimensions.
	\item\texttt{\textbf{-S ldim}}: A \textit{smart} cuboid with the largest dimension being \texttt{ldim} is evaluated. The other dimensions are proportional to the molecule's dimensions.
\end{itemize}

In the current version, \texttt{dtkcube} does not rotate the molecule to minimize the cuboid dimensions, but only looks for the highest and lower values of the atoms' coordinates in all of the three axis. These values are the ones used to adjust the cuboid.

\rule{\textwidth}{1pt}
{\center\texttt{dtkcube} help menu.\\}
\rule{\textwidth}{1pt}
\begin{footnotesize}
\begin{verbatim}
Usage:

	dtkcube wf?name [option [value(s)]] ... [option [value(s)]]

Where wf?name is the input wfx(wfn) name, and options can be:

  -l           Write cpu time, input/output information etc. on a log file
  -n  dim      Set the number of points per direction for the cube
                  to be dim x dim x dim.
  -N nx ny nz Set the individual points per direction for the cube
                  to be nx x ny x nz.
  -o outname   Set the output file name.
  -s           Use a smart cuboid for the grid. The number of points for the
                  largest direction will be 80.
  -S ln        Use a smart cuboid for the grid. ln is the number of points
                  the largest axis will have. The remaining axes will have
                  a number of points proportional to its length.
  -p prop      Choose the property to be computed. prop is a character,
                  which can be (d is the default value): 
                     d (Density)
                     g (Magnitude of the Gradient of the Density)
                     l (Laplacian of density)
                     K (Kinetic Energy Density K)
                     G (Kinetic Energy Density G)
                     E (Electron Localization Function -ELF-)
                     L (Localized Orbital Locator -LOL-)
                     M (Magnitude of the gradient of LOL)
                     S (Shannon Entropy Density)
                     V (Molecular Electrostatic Potential)
  -z           Compress the cube file using gzip (which must be intalled
                  in your system).
  -V           Displays the version of this program.
  -h	         Display the help menu.
  --help          Same as -h
  --version       Same as -V
\end{verbatim}
\end{footnotesize}
\rule{\textwidth}{1pt}
%----------------------------------------------------------------------------------------------
\section{dtkfindcp}\label{sec:dtkfindcp}
%----------------------------------------------------------------------------------------------

This program search critical points (CP) within a molecule. In version \dtkversion, the complete analysis is implemented for the electron density (ED). Partial implementation is offered for searching LOL critical points. In the future, additional fields will be implemented.

The complete search of critical points of a molecule can be found by the following command:\\
\progusg{dtkfindcp}{}
This will find all the ED critical points (including ACPs, BCPs, RCPs, and CCPs), as well as the bond gradient paths (paths that connect ACPs with BCPs). The information of the critical points are saved into two files. The \texttt{log} file contains the coordinates of all CPs found, and the field properties at such points. The \texttt{cpx} file shares the coordinates of the CPs, but additional information is stored in these files, such as the coordinates of the gradient bond paths, and the name of the wave function used to perform the search of CPs.

\texttt{dtkfindcp} is also capable of producing \texttt{pov} files (and internal callings to povray) for rendering 3D images of the critical points. For instance, the Figure  was produced by typing\\
\progusg{dtkfindcp}{-P -g -T -k}
This command produces a \texttt{pov} file and calls povray to render the image displayed in Fig. \ref{fig:dtkfindcpusex}. The option \texttt{-g} tells \texttt{dtkfindcp} to display the bond gradient paths; and the option \texttt{-T} sets the style of the bond gradient paths to be tubes.
%
\begin{figure}[hb!]
\centering
\includegraphics[width=0.45\textwidth]{cyclopCPs}%\label{fig:cyclopCPs}
\caption{Electron density critical points of the cyclopropane molecule. The green lines are the bond paths joining the BCPs and the ACPs.}\label{fig:dtkfindcpusex}
\end{figure}
%

If you are interested only in finding the critical points, but you do not want to seek the gradient bond paths, you can deactivated the search of bond paths by using the option \texttt{-G}.

If you do not wish to create a \texttt{png} image, but you do want the \texttt{pov} file, using the option \texttt{-p} will accomplish that.

In addition to the \texttt{cpx} file, you can also save the coordinates of the critical points in a very simple format. Using the option \texttt{-m}, the coordinates of the critical points will be written into three different files. The names of these files will be created out of the \texttt{wfx(wfn)} file, or out of the output file name especified by using the option \texttt{-o outputname}. Different string will be added to the out names to identify which information they contain, and the files will obey the following formats:
\begin{itemize}
   \item The file ending with ``-ATCrds.dat'' will contain the atomic coordinates, and the format of such file is\\
   \texttt{AtomicSymbol \ AtomicNumber \ X \ Y \ Z}
   \item The file ending with ``-CPCrds.dat'' will contain the coordinates of all the critical points that were found. The format of this file is\\
   \texttt{??? \ X \ Y \ Z}\\
   where \texttt{???} can be ``\texttt{acp}'', ``\texttt{bcp}'', ``\texttt{rcp}'' or ``\texttt{ccp}'' to identify the signature of the corresponding CP.
   \item The file ending with ``-BPCrds.dat'' will contain the coordinates of the points of all bond paths, without further identification. This is, the coordinates are points that belong to a bond path, but there is no way to identify to which specific bond path the point at hand belongs to. The format is simply\\
   \texttt{X \ Y \ Z}
\end{itemize}

Some of the options for calling povray can be set in the command line. For instance, the camera position can be handled by using the option \texttt{-c p}, where \texttt{p} is an integer that gives the normal vector components where the camera will be placed at. \texttt{p} can be \texttt{001,010,100,101,110,111}. Every value is a three digit number that represents the components of a vector, for instance \texttt{101} means that the camera will be placed at a position proportional to the vector (1,0,1), and so on. Warning: this option is not really compatible with the custom angle views that can be modified in the pov file (see \S\ref{sec:povcustopts} for more details), and perhaps will be deleted in future implementations.

If you wish to create a \texttt{png} image, and keep the \texttt{pov} file all at once, you need to especify the option \texttt{-k} for not deleting the pov file.

Below you can find a brief description of the \texttt{cpx} file format as well as a short overview of the custom options that can be easily modified in the pov files which, in junction with the script \texttt{dtkpov2png}, can  be used to create custom images of the displayed properties.

Finally, in future implementations there will be the possibility to search critical points of different fields. In version \dtkversion, the LOL is partially implemented. 


\rule{\textwidth}{1pt}
{\center\texttt{dtkfindcp} help menu.\\}
\rule{\textwidth}{1pt}
\begin{footnotesize}
\begin{verbatim}
Usage:

	dtkfindcp wf?name [option [value(s)]] ... [option [value(s)]]

Where wf?name is the input wfx(wfn) name, and options can be:

  -a           Draw transparent spheres around each nuclei in the wf? file.
  -o outname   Set the output file name (*.log).
                  (If not given the program will create one out of
                  the input name; if given, the pov file will
                  use this name as well --but different extension--).
  -m           Save the coordinates of atoms, critical points and gradient
                  paths (if calculated, see option -G) into *dat files.
  -p           Create a pov file. No image will be generated.
  -P           Create a png image using povray (calls povray),
                  deletes pov file.
  -c p         Set the camera vector direction for the povray scene.
                  p is an integer which can take the values
                  001,010,100,101,110,111; the camera will be located
                  somewhere in the direction (x,y,z), where xyz is one of 
                  the above values.
  -g           Draw the bond paths (gradient paths) in the POV file.
  -T           Set the style for the gradient paths to be tubes.
  -G           Skip the calculation of the bond gradient paths.
  -k           Keeps the pov file if option -P is used.
  -t cpt       Set the type of critical point to be searched. cpt can be 
                  one of the following fields (Density is the default):
                     d (Density)
                     L (Localized Orbital Locator -LOL-)
  -v           Verbose mode (displays additional information (for example the 
                  output of povray, etc.
  -V           Display the version of this program.
  -h	         Display the help menu.
  --help          Same as -h
  --version       Same as -V
********************************************************************************
  Note that the following programs must be properly installed in your system:
                        povray, graphicsmagick, gzip
********************************************************************************
\end{verbatim}
\end{footnotesize}
\rule{\textwidth}{1pt}


%..............................................................................................
\subsection{\label{sec:cpxfilefmt}The \texttt{cpx} file format}
%..............................................................................................

We follow the same conventions as for the \texttt{wfx} file format \cite{bib:webwfxformat}. A \texttt{cpx} file is somewhat similar to an XML file, but with certain restrictions. We define opening and closing tabs, for instance
%
\begin{verbatim}
<WaveFunctionFileName>
 h2o.wfx
</WaveFunctionFileName>
\end{verbatim}
or
\begin{verbatim}
<CriticalPointType>
 Electron Density
</CriticalPointType>
\end{verbatim}
or
\begin{verbatim}
<NumberOfACPs>
 3
</NumberOfACPs>
\end{verbatim}

However, any opening or closing tag MUST stand alone in a single line. The purpose of this restriction is to accelerate the reading of the \texttt{cpx} file. This format does not contain tags with spaces.

For the purposes of the data (information between tags) a new line is considered as equivalent to a space or a tab character. There can be comments between two different tags, but comments between the data is not allowed.

Below we provide an example of a \texttt{cpx} file, which is the result of a search of critical points of the water molecule.

\#\textbf{WFX (WFN) file name} from which the the wave function of the molecule was obtained.
\begin{verbatim}
<WaveFunctionFileName>
 h2o.wfx
</WaveFunctionFileName>
\end{verbatim}
\#\textbf{Type of critical points:} This refers to the type of critical points, in terms of the scalar field. This is, this tag identifies whether the critical points are electron density critical points, LOL critical points, etc.
\begin{verbatim}
<CriticalPointType>
 Electron Density
</CriticalPointType>
\end{verbatim}
\#\textbf{Number of critical points:} The total numbers of critical points by signature identification.
We follow the conventional naming:
\begin{itemize}
   \item \textit{Attractor Critical Point} \textbf{(ACP)} is a CP with signature (3,-3), and corresponds to a maximum in the topology of the Fields manifold.
   \item \textit{Bond Critical Point} \textbf{(BCP)} is a CP with signature (3,-1), and corresponds to a saddle point where two of the Hessian eigenvalues are negative, and the third is positive.
   \item \textit{Ring Critical Point} \textbf{(RCP)} is a CP with signature (3,+1), and corresponds to a saddle point where only one of the Hessian eigenvalues is negative and the other two are positive.
   \item \textit{Cage Critical Point} \textbf{(CCP)} is a CP with signature (3+3), and corresponds to a local minimum in the field's manifold.
\end{itemize}
We are aware that the terms ACP, BCP, etc. loose their physical meaning when fields other than the electron density are analysed. However, the signature values remain the same, therefore this naming can be used for uniquely tagging a critical point.
\begin{verbatim}
<NumberOfCriticalPoints>
<NumberOfACPs>
 3
</NumberOfACPs>
<NumberOfBCPs>
 2
</NumberOfBCPs>
<NumberOfRCPs>
 0
</NumberOfRCPs>
<NumberOfCCPs>
 0
</NumberOfCCPs>
</NumberOfCriticalPoints>
\end{verbatim}
\#\textbf{CP Cartesian Coordinates}: The Cartesian coordinates of all critical points with a given signature. If there is no critical point of certain signature, the opening/closing tags will appear and there will be no data among the tags.
\begin{verbatim}
<ACPCartesianCoordinates>
  7.405253164682e-34 -1.481050632936e-33  2.403142046477e-01
  0.000000000000e+00  1.432892399165e+00 -9.612568185908e-01
 -1.754787090160e-16 -1.432892399165e+00 -9.612568185908e-01
</ACPCartesianCoordinates>
<BCPCartesianCoordinates>
  3.879598339682e-19  1.056340227248e+00 -6.465448941267e-01
 -1.302800283929e-16 -1.056340227248e+00 -6.465448941267e-01
</BCPCartesianCoordinates>
<RCPCartesianCoordinates>
</RCPCartesianCoordinates>
<CCPCartesianCoordinates>
</CCPCartesianCoordinates>
\end{verbatim}
\#\textbf{Bond critical points connectivity}. This block contains the information of how the network of BCPs are related to the ACPS. This makes more sense for the electron density CPs, since for this case, a BCP is usually found whenever two atoms are bonded. For other fields, the rule of starting the search of a BCP in between every two ACPs generally applies, thus this information saves the hystory of where the seed for looking the BCP at hand was located.
\begin{verbatim}
<BCPConnectivity>
1 2 1 2
2 2 1 3
</BCPConnectivity>
\end{verbatim}
\#\textbf{Labels}: This set of tags stores the labels of all the critical points of a certain signature. For the electron density ACPs, we use the atom labels provided in the \texttt{wfx(wfn)} file. For other fields, we use also the atom label plus an extra label after the numerical index. The BCPs are labeled according to the ACPs that were used to start the search (the midpoint between each pair of ACPs). The RCPs' labeling involvels all the atoms that were related to its search. And the same for the CCPs. Once again, if the number of CPs of a certain signature is zero, the opening/closing tags will be present, but there will be no data among these tags.

It is assumed that each label consists of a set of alphanumeric characters and hyphens. However, spaces in a label are NOT allowed. In fact for this tag, a space, a tab character, or an end line, all of them indicate the final of a label.

\begin{verbatim}
<ACPLabels>
 O1 H2 H3
</ACPLabels>
<BCPLabels>
 O1-H2 O1-H3
</BCPLabels>
<RCPLabels>
</RCPLabels>
<CCPLabels>
</CCPLabels>
\end{verbatim}
\#\textbf{Number of bond paths}: Self descriptive.
\begin{verbatim}
<NumberOfBondPaths>
 2
</NumberOfBondPaths>
\end{verbatim}
\#\textbf{Number of points per bond path:} There is no \textit{a priori} way to know how many points will be per each bond path, since they will vary in length and curvature. Therefore, the number of points that belong to each bond path is set in this tag. Notice that we already know how many bond paths were found (see above, ``NumberOfBondPaths'').
\begin{verbatim}
<NumbersOfPointsPerBondPath>
 21 21
</NumbersOfPointsPerBondPath>
\end{verbatim}
\#\textbf{Bond paths' data}: This block contains the coordinates of each point that belongs to a given bond path. The data containts two child tags ``BondPathIndex'' which is a number to identify the bond path, and is given just as a redundant mechanism to check the integrity of the \texttt{cpx} file; and ``CoordinatesOfBondPathPoints'' which contains the actual coordinates of each point in the bond path identified by ``BondPathIndex''.
\begin{verbatim}
<BondPathsData>
<BondPathIndex>
 0
</BondPathIndex>
<CoordinatesOfBondPathPoints>
  1.283310240215e-18  1.325322904480e+00 -8.521158631320e-01
 -3.114719658371e-18  1.277560298804e+00 -8.844656336821e-01
  2.366717289600e-19  1.331886275170e+00 -8.749446714881e-01
  2.187505553771e-19  1.286998381318e+00 -8.383698088888e-01
  2.768727742932e-19  1.210147039589e+00 -7.743864478328e-01
  3.249714975667e-19  1.133255359299e+00 -7.104515640576e-01
  3.879598339682e-19  1.056340227248e+00 -6.465448941267e-01
  4.509481703697e-19  9.794250951975e-01 -5.826382241959e-01
  5.009611731128e-19  9.025741185897e-01 -5.186544264342e-01
  5.583774862697e-19  8.258317118277e-01 -4.545404760932e-01
  6.244561433914e-19  7.492696478497e-01 -3.902113108375e-01
  7.061472903938e-19  6.729180248149e-01 -3.256325051397e-01
  8.204186617202e-19  5.966597044907e-01 -2.609434996374e-01
  1.004820096952e-18  5.199999203721e-01 -1.967319120013e-01
  1.281126284848e-18  4.419521608625e-01 -1.342189650488e-01
  1.460104437383e-18  3.628958273391e-01 -7.298135325864e-02
  1.328768819774e-18  2.852253826867e-01 -1.000206373725e-02
  1.016984895739e-18  2.092544123305e-01  5.502203677181e-02
  6.596232316936e-19  1.340530676658e-01  1.209364863332e-01
  2.926802524295e-19  5.925397145290e-02  1.873072163197e-01
  7.405253164682e-34 -1.481050632936e-33  2.403142046477e-01
</CoordinatesOfBondPathPoints>
<BondPathIndex>
 1
</BondPathIndex>
<CoordinatesOfBondPathPoints>
  7.405253164682e-34 -1.481050632936e-33  2.403142046477e-01
 -2.771493234187e-17 -5.925397145290e-02  1.873072163197e-01
 -6.250711727050e-17 -1.340530676658e-01  1.209364863332e-01
 -9.650047009393e-17 -2.092544123305e-01  5.502203677181e-02
 -1.265815303299e-16 -2.852253826867e-01 -1.000206373725e-02
 -1.413403025249e-16 -3.628958273391e-01 -7.298135325864e-02
 -1.307349355991e-16 -4.419521608625e-01 -1.342189650488e-01
 -1.138199423562e-16 -5.199999203721e-01 -1.967319120013e-01
 -1.058905765908e-16 -5.966597044907e-01 -2.609434996374e-01
 -1.046062180849e-16 -6.729180248149e-01 -3.256325051397e-01
 -1.067544859534e-16 -7.492696478497e-01 -3.902113108375e-01
 -1.109011672163e-16 -8.258317118277e-01 -4.545404760932e-01
 -1.164060684268e-16 -9.025741185897e-01 -5.186544264342e-01
 -1.229057174906e-16 -9.794250951975e-01 -5.826382241959e-01
 -1.302800283929e-16 -1.056340227248e+00 -6.465448941267e-01
 -1.376543392952e-16 -1.133255359299e+00 -7.104515640576e-01
 -1.466480850572e-16 -1.210147039589e+00 -7.743864478328e-01
 -1.557516934400e-16 -1.286998381318e+00 -8.383698088888e-01
 -1.583008836928e-16 -1.331886275170e+00 -8.749446714881e-01
 -3.405682644852e-16 -1.277560298804e+00 -8.844656336821e-01
 -9.982455406799e-17 -1.325322904480e+00 -8.521158631320e-01
</CoordinatesOfBondPathPoints>
</BondPathsData>

\end{verbatim}

%..............................................................................................
\subsection{\label{sec:povcustopts}Options in the \texttt{pov} file}
%..............................................................................................

Every \texttt{pov} file created by \texttt{dtkfindcp} has the following header:
\begin{footnotesize}
\begin{verbatim}
#version 3.6; //Unless you know what you are doing, do not modify this line...
#include "colors.inc"
////////////////////////////////////////////////////////////////////////////////
//
//Below you can find some options to be parsed to povray
//set your custom values.
//You can reconstruct the image using the script dtkpov2png
//
////////////////////////////////////////////////////////////////////////////////
#declare GNUPlotAngle1=0;
#declare GNUPlotAngle2=0;
#declare YAngle=0;
#declare DrawAtomTranspSpheres=false;
#declare DrawStandardBonds=false;
#declare DrawAttractorCriticalPoints=true;
#declare DrawBondCriticalPoints=true;
#declare DrawRingCriticalPoints=true;
#declare DrawCageCriticalPoints=true;
#declare DrawGradientPathSpheres=false;
#declare DrawGradientPathTubes=true;
//  Activation of "DrawGradientPathSpheres" requires deactivation of 
//  "DrawGradientPathTubes", and vice versa.
#declare RadiusAllCriticalPoints=0.1;
#declare ColorACP=rgb <0.0,0.0,0.0>;
#declare RadiusACP=RadiusAllCriticalPoints;
#declare ColorBCP=rgb <0.3,0.3,0.3>;
#declare RadiusBCP=RadiusAllCriticalPoints;
#declare ColorRCP=rgb <0.6,0.6,0.6>;
#declare RadiusRCP=RadiusAllCriticalPoints;
#declare ColorCCP=rgb <0.9,0.9,0.9>;
#declare RadiusCCP=RadiusAllCriticalPoints;
#declare ColorABGradPath=rgb <0.0,1.0,0.0>;
#default { finish { specular 0.3 roughness 0.03 phong .1 } }
////////////////////////////////////////////////////////////////////////////////
//For the colors, instead of rgb <...>, you may want to try Red, Yellow, ...
//  or any of the colors defined in "colors.inc"
//
////////////////////////////////////////////////////////////////////////////////
// END OF CUSTOM OPTIONS
////////////////////////////////////////////////////////////////////////////////
\end{verbatim}
\end{footnotesize}

Getting the right position of the povray's camera is a bit tricky, and we do not attemt to solve this problem in an automatic way. However, we provide a relatively easy-to-use tool to find an adequate view. Using the program \texttt{dtkqdmol}, you can create a quick view of the molecule, and since it is rendered via gnuplot, two view angles are displayed in the window (see \S\ref{sec:dtkqdmol}). You can adjust the view through the interactive gnuplot's window and then change the values of the defined variables \texttt{GNUPlotAngle1}, and \texttt{GNUPlotAngle1}. An extra angle can be specified in povray (which gnuplot does not use): the \texttt{YAngle} variable. For more information about this view angles, see the povray's documentation (under the ``camera's rotate'' section) or the gnuplot's help menu (\texttt{help set view}).

By default, all critical points are drawn with the same size. For setting different sizes, you can adjust the options ``\texttt{RadiusACP=RadiusAllCriticalPoints;}'' to you desired value. (Changing only the value of \texttt{RadiusAllCriticalPoints} will change the size of ALL critical points.)

The colors of the critical points are given by three numbers, correspoding to the red, green or blue saturation value, and they all must be numbers between 0 and 1.

The rest of the options are self descriptive, and the only note is to use \texttt{true} or \texttt{false}.

%----------------------------------------------------------------------------------------------
\section{dtkmomd}
%----------------------------------------------------------------------------------------------

This program evaluates the electron momentum density (EMD), \textit{i.e.,} the Fourier transform of the electron density. This single program may evaluate the momentum density for a single point; one, two and three dimensional grids. The output, if any, is written as explaneid below.

\begin{itemize}
   \item \textbf{Point (0D)}: No output is created, instead the value is displayed at the screen. This grid is set by the option \texttt{-0 $\mathtt{P}_\mathtt{x}$ $\mathtt{P}_\mathtt{y}$ $\mathtt{P}_\mathtt{z}$}, where \texttt{$\mathtt{P}_\mathtt{i}$} are the values at which the EMD is evaluated.
   \item \textbf{Line (1D)}: The output is written to a \texttt{dat} file and, in the current version, only lines parallel to the axis are implemented. This grid is accessible through the option \texttt{-1 x}, where ``\texttt{x}'' indicates the axis to be evaluated and can take the values ``\texttt{x}'', ``\texttt{y}'' or ``\texttt{z}''. Creation of gnuplot scritps is enabled for this grid (option \texttt{-P}).
   \item{Plane (2D)}: The data is saved into a \texttt{tsv} file, and in the present version, only planes normal to the axis are implemented (the planes contain the origin of the momentum space coordinates). To evaluate the EMD in a plane, one must use the option \texttt{-2 xy}, where \texttt{xy} are two non-spaced characters that indicates the plane whereat the EMD will be evaluated, and can be \texttt{xy}, \texttt{xz}, or \texttt{yz}. Creation of gnuplot scritps is also enabled for this grid (option \texttt{-P}).
   \item{Cube (3D)}: The values of the EMD are stored in a \texttt{cub} file (standard Gaussian cube file), and this grid is set by using option \texttt{-3}. For this grid, there is no immediate visualization, but it needs to be produced through other programs such as VMD.
\end{itemize}
%
\begin{figure}[hb!]
\centering
\subfigure[Plane $P_x$-$P_y$]{\includegraphics[width=0.45\textwidth]{cyclopropaneMomDens-PxPy}\label{fig:cyclopMomDPxPy}}\quad
\subfigure[Plane $P_y$-$P_z$]{\includegraphics[width=0.45\textwidth]{cyclopropaneMomDens-PyPz}\label{fig:cyclopMomDPyPz}}
\caption{Evaluating the momentum density of the cyclopropane molecue on two-dimensional grids (planes).}\label{fig:dtkmomdusex}
\end{figure}
%

\rule{\textwidth}{1pt}
{\center\texttt{dtkfindcp} help menu.\\}
\rule{\textwidth}{1pt}
\begin{footnotesize}
\begin{verbatim}
Usage:

	dtkmomd wf?name [option [value(s)]] ... [option [value(s)]]

Where wf?name is the input wfx(wfn) name, and options can be:

  -0 x y z     Evaluate the momentum density at the point (x,y,z).
                  Here x,y, and z are numbers. No file is created.
  -1  x        Evaluate the momentum density on a line. "x" sets the coordinate
                  direction, and can take the values x, y or z.
                  Here x is a character. It creates a *.dat file.
  -2  xy       Evaluate the momentum density on a plane. "xy" sets the plane of
                  interest, and can take the values xy,xz,yz.
                  Here xy are two characters. It creates a *.tsv file.
  -3           Evaluate the momentum density on a cube.
                  It creates a *.cub file.
  -n  dim      Set the number of points for the cub/tsv/dat file per direction
  -o outname   Set the output file name.
                  (If not given the program will create one out of
                  the input name; if given, the dat/tsv/cub/gnp/pdf files will
                  use this name as well --but different extension--).
  -P           Create a plot using gnuplot. (Only works with options -1 or -2)
  -k           Keeps the *.gnp file to be used later by gnuplot.
  -v           Verbose (display extra information, usually output from third-
                  party sofware such as gnuplot, etc.)
  -z           Compress the tsv file using gzip (which must be intalled
                  in your system).
  -V           Displays the version of this program.
  -h           Display the help menu.
  --help          Same as -h
  --version       Same as -V
********************************************************************************
  Note that the following programs must be properly installed in your system:
                      gnuplot, epstool, epstopdf, gzip
********************************************************************************
\end{verbatim}
\end{footnotesize}
\rule{\textwidth}{1pt}



%----------------------------------------------------------------------------------------------
\section{dtkdemat1}
%----------------------------------------------------------------------------------------------

This program evaluates the density matrix of order 1 (DM1) along a line, which can be the line that joins two atoms, or the bond path that connects those two ACPs. In this version, only the electron density bond path can be selected. 

\texttt{dtkdemat1} can be called, for example, by\\
\progusg{dtkdemat1}{-P -l -c -s 0.02}

The above command was used for evaluating the data shown in Fig. \ref{fig:dtkdemat1usex}. The output of the program consists on four files. The data for generating the three-dimensional and two-dimensional plots is saved in a \texttt{tsv} file. The data for the main and secondary diagonal (white solid and dashed line of figure \ref{fig:md12d}) are saved in two \texttt{dat} files. In addition, a \texttt{log} file is created, where some information about the position of the critical point (or the minimum of the ED) is located, etc.
%
\begin{figure}[hb!]
\centering
\subfigure[DM1 (3D)]{\includegraphics[width=0.55\textwidth]{cyclopropaneBPDM1-C1-C2-3D}\label{fig:md13d}}\quad
\subfigure[DM1 (2D)]{\includegraphics[width=0.35\textwidth]{cyclopropaneBPDM1-C1-C2-2D}\label{fig:md12d}}\\
\subfigure[DM1 (1D1) along the main diagonal (white solid line in \subref{fig:md12d}).]{\includegraphics[width=0.4\textwidth]{cyclopropaneBPDM1-C1-C2-1D1}\label{fig:md11d1}}\quad
\subfigure[DM1 (1D2) along the secondary diagonal (white dashed line in \subref{fig:md12d}).]{\includegraphics[width=0.4\textwidth]{cyclopropaneBPDM1-C1-C2-1D2}\label{fig:md11d2}}\\
\caption{Evaluating the density matrix of order 1 with \texttt{dtkdemat1}.}\label{fig:dtkdemat1usex}
\end{figure}
%

As usual, the option \texttt{-P} requests the creation of plots. This program generates four of them. The names of the plots use the input wave function file name, adds the characters \texttt{BP} (if the bond path is selected), or \texttt{SL} if the straight line that joins two atoms (especified with the option \texttt{-a $\mathtt{a}_1$ $\mathtt{a}_2$}, see below), the labels of the selected atoms, and two or three characters to indicate the specific data displayed in the plot. The DM1 information is displayed as a surface (its name ends with \texttt{3D}), as a heat map (whose name ends with \texttt{2D}), and two simple plots that contains the DM1 values at the main diagonal (solid white line in Figure \ref{fig:md12d}, and whose name ends with \texttt{1D1}) and at the secondary diagonal (dashed white line in Figure \ref{fig:md12d}, and whose name ends with \texttt{1D2}).

The option \texttt{-l} makes \texttt{dtkdemat1} to draw the labels of the atoms for the two dimensional and one-dimensional plots, while the option \texttt{-c} draws the contour lines as shown, for example, in Figure \ref{fig:md12d}.

By default, \texttt{dtkdemat1} uses the first and second atoms of the \texttt{wfx/wfn} file. For selecting different atoms, you should use the option \texttt{-a $\mathtt{a}_1$ $\mathtt{a}_2$}, where \texttt{a}$_1$, and \texttt{a}$_2$ are the desired atoms.

As well, by default, the program will calculate the bond path between the atoms, and then use this curve to evaluate the DM1. However, in some cases is needed to evaluate the DM1 along the straight line that joins the atoms, which can be requested with the switch \texttt{-L}.

For increasing the number of points within the bond path line, you may want to look for a combination of the option \texttt{-s step} and \texttt{-n dim}. Here \texttt{step} orders \texttt{dtkdemat} to use the step \texttt{step} for the searching of the bond path (this is the step used in the Runge-Kutta integrator), while \texttt{dim} is the dimension of an array that contains the maximum number of points contained in the bond path. By default, \texttt{dtkdemat1} sets \texttt{step=0.03}, and \texttt{dim=200}. When requesting to evaluate DM1 over the line that joins the atoms, \texttt{step} is no longer significant, rather the line will contain \texttt{dim} points. 


\rule{\textwidth}{1pt}
{\center\texttt{dtkfindcp} help menu.\\}
\rule{\textwidth}{1pt}
\begin{footnotesize}
\begin{verbatim}
Usage:

	dtkdemat1 wf?name [option [value(s)]] ... [option [value(s)]]

Where wf?name is the input wfx(wfn) name, and options can be:

  -a a1 a2     Define the atoms  (a1,a2) used to define bond path/line.
                  If this option is not activated, the program will 
                  set a1=1, a2=2.
                  Note: if the *.wfn (*.wfx) file has only one atom
                  the program will exit and no output will be generated.
  -L           Calculate DM1 upon the straight line that joins the atoms
                  instead of upon the bond path.
  -n  dim      Set the number of points for the tsv file per direction.
                  Note: for the bond path you may want to look for a good 
                  combination of n and the number "step" given in option -s,
                  since the number of points in the bond path will be mainly 
                  governed by step.
  -o outname   Set the output file name.
                  (If not given the program will create one out of
                  the input name; if given, the tsv, gnp and pdf files will
                  use this name as well --but different extension--).
  -s step      Set the stepsize for the bond path to be 'step'.
                  Default value: 0.03
  -P           Create a plot using gnuplot.
  -c           Show contour lines in the plot.
  -l           Show labels of atoms (those set in option -a) in the plot.
  -v           Verbose (display extra information, usually output from third-
                  party sofware such as gnuplot, etc.)
  -z           Compress the tsv file using gzip (which must be intalled
                  in your system).
  -h           Display the help menu.
  -V           Displays the version of this program.
  --help          Same as -h
  --version       Same as -V

********************************************************************************
  Note that the following programs must be properly installed in your system:
                     gnuplot, epstool, epstopdf, gzip
********************************************************************************
\end{verbatim}
\end{footnotesize}
\rule{\textwidth}{1pt}

%----------------------------------------------------------------------------------------------
\section{dtkbpdens}
%----------------------------------------------------------------------------------------------

This program seeks the bond path between two atoms and then evaluate the requested scalar/vector field at the points belonging to the bond path. This is specially meaningfull when the bond path is not the straight line that joins the atoms, and especially in cases where they differ considerable such as in hydrogen bonds.

The data for producing Figure \ref{fig:dtkbpdensusex} was obtained by typing the two following commands.\\
\phantom{M}\\
\texttt{\phantom{MMM}\$dtkbpdens ciclopropano\_pbe6311.wfx -P -l -s 0.02 -p M}\\\phantom{M}\\
\texttt{\phantom{MMM}\$dtkbpdens ciclopropano\_pbe6311.wfx -P -l -n 200 -p M -L}\\\phantom{M}\\
%
\begin{figure}[ht!]
\centering
\includegraphics[width=0.45\textwidth]{cyclopropaneBPSLMagGradLOL-C1-C2}
\caption{Evaluating the magnitude of LOL along the bond path}\label{fig:dtkbpdensusex}
\end{figure}
%

Plots will be generated by using option \texttt{-P}, and the labels of the atoms will be added activating the switch \texttt{-l}.

The step for the Runge-Kutta integrator can be set with the option \texttt{s step}, and the maximum number of points for the bond path is set with option \texttt{-n dim}. Here \texttt{dim} is the number of points requested for the arrays that contains the coordinates of the points belonging to the bond path.

If the property is evaluated at the line that join the atoms (activating the switch \texttt{-L}), then the variable \texttt{step} is no longer used (whose default value is \texttt{step=0.03}), and instead the value of \texttt{dim} is used to divide the line (the default value for this variable is \texttt{dim=200}). The output will differ from the data obtained with \texttt{dtkline} in the length of the line: \texttt{dtkbpdens} evaluates a field between the atoms, while \texttt{dtkline} evaluates the property upon a longer line that passes through the same atoms.

To choose the scalar field to plot, one simply uses the option \texttt{-p c}, where \texttt{c} is a character that can be one of those included in the list shown in the menu (below).

\rule{\textwidth}{1pt}
{\center\texttt{dtkfindcp} help menu.\\}
\rule{\textwidth}{1pt}
\begin{footnotesize}
\begin{verbatim}
Usage:

	dtkbpdens wf?name [option [value(s)]] ... [option [value(s)]]

Where wf?name is the input wfx(wfn) name, and options can be:

  -a a1 a2     Define the atoms  (a1,a2) used to define bond path/line.
                  If this option is not activated, the program will 
                  set a1=1, a2=2.
                  Note: if the *.wfn (*.wfx) file has only one atom
                  the program will exit and no output will be generated.
  -L           Calculate the field upon the straight line that joins the atoms
                  instead of upon the bond path.
  -n  dim      Set the number of points for the dat file.
                  Note: for the bond path you may want to look for a good 
                  combination of n and the number "step" given in option -s,
                  since the number of points in the bond path will be mainly 
                  governed by step.
  -o outname   Set the output file name.
                  (If not given the program will create one out of
                  the input name; if given, the dat, gnp and (eps)pdf files will
                  use this name as well --but different extension--).
  -p prop      Choose the property to be computed. prop is a character,
                  which can be (d is the default value): 
                     d (Density)
                     g (Magnitude of the Gradient of the Density)
                     l (Laplacian of density)
                     K (Kinetic Energy Density K)
                     G (Kinetic Energy Density G)
                     E (Electron Localization Function -ELF-)
                     L (Localized Orbital Locator -LOL-)
                     M (Magnitude of the Gradient of LOL)
                     S (Shannon Entropy Density)
                     V (Molecular Electrostatic Potential)
  -s step      Set the stepsize for the bond path to be 'step'.
                     Default value: 0.03
  -P           Create a plot using gnuplot.
  -l           Show labels of atoms (those set in option -a) in the plot.
  -v           Verbose (display extra information, usually output from third-
                  party sofware such as gnuplot, etc.)
  -z           Compress the dat file using gzip (which must be intalled
                  in your system).
  -h           Display the help menu.
  -V           Displays the version of this program.
  --help          Same as -h
  --version       Same as -V

--------------------------------------------------------------------------------
         The format of the dat file is:
            L  X  Y  Z  V
         where L is the value of the parameter that maps the bond
         path to a line; X, Y, and Z are the actual spatial coordinates
         of each point in the bond path; and V is the value of the
         chosen field at the point (X,Y,Z) ---see option -p.
--------------------------------------------------------------------------------
********************************************************************************
  Note that the following programs must be properly installed in your system:
                           gnuplot, epstopdf, gzip
********************************************************************************
\end{verbatim}
\end{footnotesize}
\rule{\textwidth}{1pt}

As we can see in the help menu, the output file has a distinctive format. It contains the value of the variable used for the parametrization of the bond path (or straight line), the actual coordinates in the Cartesian system wherein the molecule is actually embedded, and finally the requested field.

%----------------------------------------------------------------------------------------------
\section{\label{sec:dtkqdmol}dtkqdmol}
%----------------------------------------------------------------------------------------------

The purpose of this program is to provide a quick view of the molecule through the gnuplot interactive terminal. This is particularly useful for viewing the labels of the atoms, and to produce simple diagrams, such as the Figures \ref{fig:cyclopmolline} and \ref{fig:cyclopmolplane}.

\texttt{dtkqdmol} is \textit{not} intended to replace more powerful visualization tools such as JMol, or the viewers included in the \textit{ab initio} calculation programs, but to serve as a command line program to improve the workflow of analysing and evaluating properties, and also to adjust the view angles to be parsed to povray (see \texttt{dtkfindcp} section). In fact the name is a mnemonic name from \texttt{dtk quick draw molecule}.

The visualization of the molecule is carried out using the stable and powerful interactive terminals of gnuplot. Usually the standard terminal under Unix-like and MacOSX systems is \texttt{x11} (chosen by default by \texttt{dtkqdmol}), and \texttt{windows} under Windows (also default when \texttt{dtkqdmol} is compiled with cygwin). However, one can especify a custom terminal by using the option \texttt{-t term}, where \texttt{term} is the desired terminal. We use quite regularly the terminals \texttt{x11} and \texttt{qt}; please ensure that \texttt{term} is working properly in your system and that is recognized by gnuplot.

The following command line should create a \texttt{gnp} file and immediately call gnuplot, which in turn opens an interactive window with the cyclopropane molecule. A snapshot of the opened window is shown in Fig. \ref{fig:dtkqdmolusex}\\
\progusg{dtkqdmol}{-t qt -r}
The option \texttt{-t qt} only works when the Qt terminal is installed in your system and propperly linked to gnuplot.

To immediately call gnuplot, use the option \texttt{-r}. If this switch is not activated, the program will create the \texttt{gnp} file, which can be run later with gnuplot.
%
\begin{figure}[hb!]
\centering
\includegraphics[width=0.5\textwidth]{dtkqdmolusex}
\caption{Snapshot of the interactive terminal qt under MacOSX.}\label{fig:dtkqdmolusex}
\end{figure}
%

To close the interactive window, select it (if it is not active), and press any key.

\rule{\textwidth}{1pt}
{\center\texttt{dtkfindcp} help menu.\\}
\rule{\textwidth}{1pt}
\begin{footnotesize}
\begin{verbatim}
Usage:

	dtkqdmol wf?name [option [value(s)]] ... [option [value(s)]]

Where wf?name is the input wfx(wfn) name, and options can be:

  -o outname   Set the output file name.
                  (If not given the program will create one out of
                  the input name; if given, the gnp file will
                  use this name as well --but different extension--).
  -r           Run the generated gnuplot file.
  -t term      Set the terminal for gnuplot (x11,qt,wxt,...)
                  (Default terminal: x11).
  -V           Displays the version of this program.
  -h           Display the help menu.
  --help          Same as -h
  --version       Same as -V
********************************************************************************
  Note that the following programs must be properly installed in your system:
                                    gnuplot
********************************************************************************
\end{verbatim}
\end{footnotesize}
\rule{\textwidth}{1pt}


%**********************************************************************************************
%**********************************************************************************************
\chapter{Scripts}
%**********************************************************************************************
%**********************************************************************************************

There are several tasks which are highly repetitive during the processing of data. \DTK{} provides some scripts that perform very simple, but multicommand, tasks. For instance, it is often necessary to correct the bounding box of \texttt{eps} figures (a task done by \texttt{epstool}) and then produce a \texttt{pdf} file (which is done by \texttt{epstopdf}). Another often task is to render a \texttt{png} image from a \texttt{pov} file, especially when adjusting the view by setting the custom options (see \S\ref{sec:dtkfindcp}). The latter task can be easily executed with the script \texttt{dtkpov2png}. Below is the sintax and some options that can be used.

For the scripts, please note that the options must be written \textit{before} the name of the source file.

%----------------------------------------------------------------------------------------------
\section{dtkeps2pdf}
%----------------------------------------------------------------------------------------------

This script has only one option for setting the output name.\\
\rule{\textwidth}{1pt}
{\center\texttt{dtkfindcp} help menu.\\}
\rule{\textwidth}{1pt}
\begin{footnotesize}
\begin{verbatim}
usage: dtkeps2pdf [option(s) [argument(s)]] [inputname.eps]

This script takes the file inputname.eps, calls epstool to correct the bounding
box of the eps, and finally calls epstopdf. The options can be:

   -o outname.pdf   Set the final name for the pdf to be "outname.pdf"
   -h               Display the help menu.

Notice that epstool and epstopdf must be installed in your system. Please, visit

http://pages.cs.wisc.edu/~ghost/gsview/epstool.htm
http://www.ctan.org/pkg/epstopdf

for more information about this excelent tools.
\end{verbatim}
\end{footnotesize}
\rule{\textwidth}{1pt}

%----------------------------------------------------------------------------------------------
\section{dtkpov2png}
%----------------------------------------------------------------------------------------------

This script has two options, one for setting the output name, and one another to set the width of the final \texttt{png} image. The height of the image will be calculated from the width, always using a ration 4:3. Hence, it is recomended that the width is a multiple of 12.

\rule{\textwidth}{1pt}
{\center\texttt{dtkfindcp} help menu.\\}
\rule{\textwidth}{1pt}
\begin{footnotesize}
\begin{verbatim}
usage: dtkpov2png [option(s) [argument(s)]] [inputname.pov]

This script takes the file inputname.pov, calls povray to create a png, then it
calls gm convert to trim the png file. The options can be:

   -o outname.png   Set the final name for the png to be "outname.png"
   -w width         Set the width of the image to be "width"
                      (Default value: 1200. Since the image will be trimmed, 
                       the actual width will be smaller than "width".)
   -h               Display the help menu.

Notice that povray and graphics magic must be installed in your system.
Please, visit

http://www.povray.org
http://www.graphicsmagick.org

for more information about this fabulous programs.
\end{verbatim}
\end{footnotesize}
\rule{\textwidth}{1pt}

\textbf{Tip:} Use a small value of width when testing the view, and use the final resolution once you have chosen the final visualization options.


