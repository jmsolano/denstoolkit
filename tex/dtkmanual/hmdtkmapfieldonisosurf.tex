                         :-) Created by: JMSA/JMHP (-:

********************************************************************************


Usage:

	dtkmapfieldonisosurf wf?name [option [value(s)]] ... [option [value(s)]]

Where wf?name is the input wfx(wfn) name, and options can be:
Where wf?name is the input wfx(wfn) name, and options can be:

  -a aG1 aG2	Set the gnuplot angles to be aG1 and aG2.
            	  Use dtkqdmol to check and set these angles.
  -A aX aY aZ	Set the view angles to be aX, aY, and aZ.
  -b val    	Set the color scale to be [-val,val].
  -B vMin vMax	Set the color scale to be [vMin,vMax].
  -c        	Skip the cube calculation. Notice this will assume that the
            	  cube was previously computed, and it is present in the
            	  present working directory.
  -i iprop   	Set the property to build the isosurface to be iprop.
            	  Here, iprop is a character that can be (z is the default):
         		d (Electron Density --Rho--)
         		g (Magnitude of the Gradient of the Electron Density)
         		l (Laplacian of Electron Density)
         		K (Kinetic Energy Density K)
         		G (Kinetic Energy Density G)
         		e (Ellipticity)
         		E (Electron Localization Function --ELF--)
         		L (Localized Orbital Locator --LOL--)
         		M (Magnitude of the Gradrient of LOL)
         		P (Magnitude of Localized Electrons Detector)
         		r (Region of Slow Electrons --RoSE--)
         		s (Reduced Density Gradient --s--)
         		S (Shannon-Entropy Density)
         		V (Molecular Electrostatic Potential)
         		D (Density Overlap Regions Indicator --DORI--)
         		q (One electron disequilibrium --Rho squared--)
         		u (Scalar Custom Field)
         		U (Vector Custom Field)
         		v (Virial Potential Energy Density)
         		z (Non Covalent Interactions - Reduced Density Gradient)
         		Z (Non Covalent Interactions - Density)
  -I isoval 	Set the isovalue of s to be isoval
  -J        	Setup alpha- and beta-spin density matrices.
  -k        	Keep the pov-ray script (see also option P, below).
  -l palette	Select the color scheme 'palette', which can be any of:
            	  bentcoolwarm blues bugn gnbu greens greys inferno
            	  magma moreland oranges orrd plasma pubu purples rdbu
            	  rdylbu rdylgn reds spectral viridis ylgn ylgnbu
            	  ylorbr ylorrd
  -m mprop  	Map the property mprop onto the isosurface. Here,
            	  mprop is a character that can be one of the fields
            	  described in option '-i', above (however the default
            	  for this option is Z). In addition, if you want to ONLY draw
            	  the isosurface, but not to map a field onto it, then
            	  set mprop to be 0.
  -n  dim   	Set the number of points per direction for the cube
            	  to be dim x dim x dim. (see dtkcube's option '-n').
  -N nx ny nz	Set the individual points per direction for the cube
            	  to be nx x ny x nz (see also dtkcube's option '-N').
  -o outname	Set the output file names to use 'outname' as a basename,
            	  i.e., the pov and png files will be named:
            	  outnameXXX.pov and outnameXXX.png, respectively. Here
            	  XXX = NCI, DORI or another descriptive label.
  -O a b c  	Orient the POV camera, so that the atoms a, b, and c (numbering
            	  according to the wf? file) are placed over the screen. The
            	  final view should look like Scheme 1, below.
  -r val    	Set the grayscale to be val. This option is useful together
            	  with option "-m 0". The isosurface will be colored using
            	  grayscale [0-1]. val=1 means white, and val=0 means black.
  -R r g b  	Set the rgb components to be r, g, and b. This option is useful together
            	  with option "-m 0". The isosurface will be colored using
            	  the rgb color generated by r, g, and b. Here r, g, and b, can
            	  take the values from 0.0 to 1.0
  -s        	Use a smart cuboid for the cube. The number of points for the
            	  largest direction will be 80
            	  (see also dtkcube's option '-s').
  -S ln     	Use a smart cuboid for the cube. ln is the number of points
            	  the largest axis will have. The remaining axes will have
            	  a number of points proportional to the molecule dimensions.
            	  (see also dtkcube's option '-S').
  -P        	Generate a pov-ray script and renders it. Notice: this requires
            	   povray to be installed in your system.
  -x alpha  	Rotate the final view by alpha degrees around the x-axis.
  -y beta   	Rotate the final view by beta  degrees around the y-axis.
  -z gamma  	Rotate the final view by gamma degrees around the z-axis.
  -V        	Display the version of this program.
  -h		Display the help menu.

  --nci      		Select the appropriate fields to render an NCI plot.
  --configure-nci rMin rMax sMax 	Set the parameters rhoMin, rhoMax,
             		  and redDensGradMax to be rMin, rMax, and sMax,
             		  respectively. Default values: rhoMin=0.0005,
             		  rhoMax=0.065, and redGradMax=2.
  --dori     		Select the appropriate fields to render a DORI map.
  --help    		Same as -h
  --version 		Same as -V
--------------------------------------------------------------------------------
            	           a
            	           |
            	          y|
            	           |--------c
            	           |________|______
            	           / b      x       
            	          /
            	       z / 
  Scheme 1: View of the aligned atoms (see option -O).
--------------------------------------------------------------------------------
